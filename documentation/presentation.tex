\documentclass{beamer}
\usepackage[utf8]{inputenc}
\usepackage[greek]{babel}
\usepackage{graphicx}
\usetheme{Madrid}

\newcommand{\tl}[1]{\textlatin{#1}}
\newcommand{\place}[1]{images/#1.png}
\setbeamertemplate{headline}{}



\title{Ανάλυση Πιστωτικού Κινδύνου με Μηχανική Μάθηση}
\author{Κατσιγιάννης Θεόφιλος}
\date{Ιούνιος 2025}

\begin{document}

\frame{\titlepage}
% Slide 2 - Περιεχόμενα
\begin{frame}{Περιεχόμενα}
\tableofcontents
\end{frame}

\section{Εισαγωγή}
% Slide 1
\begin{frame}{Εισαγωγή}
\begin{itemize}
  \item Ανάλυση τραπεζικού κινδύνου
  \item Σύγκριση μοντέλων
  \item \tl{Dataset} από \tl{Kaggle}: \tl{Credit Score}
\end{itemize}
\end{frame}
% Slide 2
\begin{frame}{Το \tl{Dataset}}
Στο πλαίσιο της παρούσας εργασίας χρησιμοποιήθηκε το σύνολο δεδομένων \textlatin{"Credit Score"} που είναι διαθέσιμο μέσω της πλατφόρμας \textlatin{Kaggle (Conor Sully, 2023)}.

\vspace{0.3cm}
Το σύνολο περιλαμβάνει 1.000 παρατηρήσεις και 87 χαρακτηριστικά, τα οποία αφορούν:
\begin{itemize}
  \item δημογραφικά και οικονομικά δεδομένα
  \item δείκτες συμπεριφοράς καταναλωτή
  \item ποσοτικά \textlatin{ratios} (π.χ. \textlatin{R\_HEALTH\_SAVINGS, R\_ENTERTAINMENT, R\_TAX\_DEBT})
\end{itemize}

\vspace{0.3cm}
Η μεταβλητή στόχος είναι η \textlatin{DEFAULT}, δυαδική: (0 = χωρίς αθέτηση, 1 = αθέτηση πληρωμής).
\end{frame}
\begin{frame}{Το \tl{Dataset}}
Το \textlatin{dataset} είναι συνθετικό αλλά ρεαλιστικό και κατάλληλο για ταξινόμηση, επιλογή χαρακτηριστικών και τεχνικές όπως:
\textlatin{CatBoost, SHAP, t-SNE}.

\vspace{0.2cm}
Η εργασία βασίζεται σε συνδυασμό αυτών μεθόδων σε συσχέτιση με το \textlatin{paper}.
\end{frame}
\section{Η Βασική Πειραματική Δομή}
% Slide 7
\begin{frame}{\tl{Pipeline} - Γενική Δομή}
\begin{itemize}
  \item Προεπεξεργασία \(\to\) Επιλογή χαρακτηριστικών \(\to\) Υποβιβασμός διάστασης \(\to\) Μοντέλο
\begin{figure}[H]
\centering
\includegraphics[scale=0.2]{\place{graph1}}
\caption{Η γενική μορφή της ακολουθίας μάθησης}
\end{figure}
\end{itemize}
\end{frame}


\section{Βασικά Εργαλεία-Δομές Αλγορίθμων}

\begin{frame}{\tl{Pearson-Correlation Analysis}}
Η \textlatin{Pearson-Correlation Analysis} χρησιμοποιεί τον δείκτη \textlatin{Pearson} για να παραλείψει χαρακτηριστικά (\textlatin{feature selection}) χωρίς γραμμική σχέση με τη στήλη στόχο.

\vspace{0.2cm}
Αποτελεί βασική τεχνική προεπεξεργασίας,για βελτιωμένο \textlatin{feature selection}.
\end{frame}

\begin{frame}{\tl{Manifold Learning} – Μέρος Α}
\begin{description}
 \item[\tl{PCA}] Γραμμική μείωση διαστάσεων. Συμπύκνωση πληροφορίας με βάση τις κύριες ιδιοτιμές του πίνακα συνδιακύμανσης.
 \item[\tl{Isomap}] Μη γραμμική μείωση με διατήρηση γεωδαιτικών αποστάσεων στον χώρο χαμηλών διαστάσεων (\textlatin{manifold}).
 \item[\tl{LLE}] Διατηρεί την τοπική γεωμετρία. Παρόμοιος με τον \textlatin{Isomap}.
\end{description}
\end{frame}

\begin{frame}{\tl{Manifold Learning} – Μέρος Β}
\begin{description}
 \item[\tl{UMAP}] Διατηρεί τοπική και σφαιρική δομή. Υποθέτει πολλαπλότητα και προσπαθεί να χαρτογραφήσει τη γεωμετρία και σχέσεις γειτονίας.
 \item[\tl{t-SNE}] Για οπτικοποίηση σε 2–3 διαστάσεις. Προσπαθεί να διατηρήσει τις τοπικές σχέσεις.
 \item[\tl{Autoencoders}] Νευρωνικά δίκτυα για υποβιβασμό διάστασης. Ο \textlatin{encoder} χαρτογραφεί σε \textlatin{latent space}, ο \textlatin{decoder} αποκαθιστά την είσοδο.
\end{description}
\end{frame}

\begin{frame}{Αλγόριθμοι Μηχανικής Μάθησης}
\begin{description}
 \item[\tl{CatBoost}] \textlatin{Boosted decision trees}. Χαρακτηριστικά:
 \begin{itemize}
  \item Αυτόματη διαχείριση κατηγορικών μεταβλητών
  \item Υψηλή ακρίβεια, χαμηλή υπερπροσαρμογή
  \item Υποστήριξη \textlatin{CPU/GPU}
  \item Ελάχιστη παραμετρικοποίηση
  \item Ιδανικός για \textlatin{classification/regression}
 \end{itemize}
 \item[\tl{Logistic Regression}] Υπολογίζει πιθανότητες ανήκειν σε κατηγορία: αν \(p > 0.5 \rightarrow 1\), αλλιώς \( \rightarrow 0 \)
\end{description}
\end{frame}

\begin{frame}{Βοηθητικά Εργαλεία}
\begin{description}
 \item[\tl{SMOTE}] Για \textlatin{class imbalance}. Δημιουργεί τεχνητά δείγματα μειοψηφικής κατηγορίας με παρεμβολή μεταξύ γειτόνων.
 \item[\tl{SHAP}] Για ερμηνεία προβλέψεων. Χρησιμοποιεί \textlatin{Shapley values} για να ποσοτικοποιήσει τη συμβολή κάθε χαρακτηριστικού.
\end{description}
\end{frame}
\section{Συνοπτικά αποτελέσματα πειραμάτων}
\begin{frame}{Συνοπτικά αποτελέσματα πειραμάτων}
\begin{table}[ht]
\centering
\resizebox{0.8\textwidth}{!}{%
\begin{tabular}{l|c|c|c|c|c|c|c}
\textbf{Μοντέλο} & \textbf{\textlatin{Accuracy}} & \textbf{\textlatin{Prec}. 0} & \textbf{\textlatin{Recall} 0} & \textbf{\textlatin{F1} 0} & \textbf{\textlatin{Prec}. 1} & \textbf{\textlatin{Recall} 1} & \textbf{\textlatin{F1} 1} \\
\hline
\textlatin{CatBoost} & 0.603 & 0.75 & 0.67 & 0.71 & 0.34 & 0.44 & 0.38 \\
\textlatin{UMAP,Logistic} & 0.613 & 0.79 & 0.63 & 0.70 & 0.38 & 0.57 & 0.45 \\
\textlatin{UMAP, CatBoost} & 0.650 & 0.79 & 0.70 & 0.74 & 0.40 & 0.52 & 0.46 \\
\textlatin{Isomap, CatBoost} & 0.643 & 0.79 & 0.69 & 0.74 & 0.39 & 0.51 & 0.45 \\
\textlatin{LLE, CatBoost} & 0.613 & 0.78 & 0.64 & 0.70 & 0.37 & 0.55 & 0.44 \\
\textlatin{PCA, CatBoost} & 0.647 & 0.83 & 0.64 & 0.72 & 0.42 & 0.65 & 0.51 \\
\textlatin{Autoencoder, CatBoost} & 0.630 & 0.78 & 0.67 & 0.72 & 0.38 & 0.52 & 0.44 \\
\end{tabular}%
}
\caption{Συγκριτικός πίνακας αποτελεσμάτων μοντέλων}
\end{table}
\begin{itemize}
  \item \tl{PCA+CatBoost}: καλύτερη ισορροπία
  \item \tl{UMAP+CatBoost}: υψηλό \tl{accuracy}
\end{itemize}
\end{frame}
\section{Το \tl{paper}}
\begin{frame}{Βιβλιογραφική Αναφορά}
\small
\begin{itemize}
  \item \textbf{Τίτλος:} \\
  \textit{\tl{A hybrid machine learning framework by incorporating categorical boosting and manifold learning for financial analysis}}

  \item \textbf{Συγγραφείς:} \tl{Yuyang Zhao} και \tl{Hongbo Zhao}

  \item \textbf{Περιοδικό:} \textit{\tl{Intelligent Systems with Applications}}, Τόμος 25, Σελίδα 200473

  \item \textbf{Έτος:} 2025

  \item \textbf{Εκδότης:} \tl{Elsevier}

  \item \textbf{Διαθέσιμο \tl{online}:} 27 Δεκεμβρίου 2024

\end{itemize}
\end{frame}

\section{Η βασική δομή του πειράματος}
\begin{frame}
\begin{figure}[H]
\centering
\includegraphics[scale=0.2]{\place{graph2}}
\caption{Η γενική μορφή της ακολουθίας μάθησης}
\end{figure}
\end{frame}

\section{Συνολικά Αποτελέσματα}
\begin{frame}{Συνοπτικά αποτελέσματα πειραμάτων}
\begin{table}[ht]
\centering
\resizebox{0.8\textwidth}{!}{%
\begin{tabular}{l|c|c|c|c|c|c|c}
\textbf{Μοντέλο} & \textbf{\textlatin{Accuracy}} & \textbf{\textlatin{Prec}. 0} & \textbf{\textlatin{Recall} 0} & \textbf{\textlatin{F1} 0} & \textbf{\textlatin{Prec}. 1} & \textbf{\textlatin{Recall} 1} & \textbf{\textlatin{F1} 1} \\
\hline
\textlatin{CatBoost} & 0.603 & 0.75 & 0.67 & 0.71 & 0.34 & 0.44 & 0.38 \\
\textlatin{UMAP,Logistic} & 0.613 & 0.79 & 0.63 & 0.70 & 0.38 & 0.57 & 0.45 \\
\textlatin{UMAP, CatBoost} & 0.650 & 0.79 & 0.70 & 0.74 & 0.40 & 0.52 & 0.46 \\
\textlatin{Isomap, CatBoost} & 0.643 & 0.79 & 0.69 & 0.74 & 0.39 & 0.51 & 0.45 \\
\textlatin{LLE, CatBoost} & 0.613 & 0.78 & 0.64 & 0.70 & 0.37 & 0.55 & 0.44 \\
\textlatin{PCA, CatBoost} & 0.647 & 0.83 & 0.64 & 0.72 & 0.42 & 0.65 & 0.51 \\
\textlatin{Autoencoder, CatBoost} & 0.630 & 0.78 & 0.67 & 0.72 & 0.38 & 0.52 & 0.44 \\
\textbf{\textlatin{Paper (CatBoost, t-SNE)}} & \textbf{0.720} & \textbf{0.77} & \textbf{0.88} & \textbf{0.82} & \textbf{0.47} & \textbf{0.28} & \textbf{0.35}
\end{tabular}%
}
\caption{Συγκριτικός πίνακας αποτελεσμάτων μοντέλων}
\end{table}
\end{frame}
\begin{frame}{Τα αποτελέσματα από \tl{t-SNE}}
\begin{figure}[H]
\centering
\includegraphics[scale=0.25]{\place{Figure_10}}
\end{figure}
Η απεικόνιση \tl{t-SNE} των δειγμάτων δείχνει ότι τα δείγματα της μειοψηφικής κατηγορίας (\tl{Default = 1}, κόκκινο) δεν σχηματίζουν ευδιάκριτα συμπλέγματα, αλλά εμφανίζονται εντός περιοχών που κυριαρχούνται από την πλειοψηφική κατηγορία. Αυτό εξηγεί τις δυσκολίες στην ταξινόμηση με χρήση γραμμικών ή δέντρων-βασισμένων μοντέλων, και υποδηλώνει την ανάγκη για ενίσχυση ή επαναδειγματοληψία της κλάσης 1.
\end{frame}
\begin{frame}{\tl{ROC curve}}
\begin{figure}[H]
\centering
\includegraphics[scale=0.3]{\place{Figure_12}}
\caption{\tl{ROC curve}}
\end{figure}
Από το παραπάνω γράφημα, ο ταξινομητής έχει περιορισμένη ικανότητα διάκρισης μεταξύ \tl{good} και \tl{bad loaners}.
\end{frame}
\begin{frame}{\tl{Precision vs Recall}}
Από το παραπάνω γράφημα, ο ταξινομητής έχει περιορισμένη ικανότητα διάκρισης μεταξύ \tl{good} και \tl{bad loaners}.
\begin{figure}[H]
\centering
\includegraphics[scale=0.3]{\place{Figure_13}}
\caption{\tl{precision-recall curve}}
\end{figure}
Από το παραπάνω δίαγραμμα,το μοντέλο έχει δυσκολία να προβλέψει τα \tl{defaults} με αξιοπρεπή \tl{recall} χωρίς να θυσιάζει την ακρίβεια. Αυτό επιβεβαιώνει την ασυμμετρία και πολυπλοκότητα του προβλήματος.
\end{frame}

\begin{frame}{\tl{SHAP Diagram}}
\begin{figure}[H]
\centering
\includegraphics[scale=0.3]{\place{Figure_15}}
\caption{Γράφημα \tl{SHAP}}
\end{figure}
\end{frame}

\begin{frame}{\tl{SHAP Diagram}}
\begin{figure}[H]
\centering
\includegraphics[scale=0.3]{\place{Figure_16}}
\caption{Γράφημα \tl{SHAP}}
\end{figure}
\end{frame}

\section{Γενικά Συμπεράσματα}
% Slide 20
\begin{frame}{Συμπεράσματα}
\begin{itemize}
  \item Χρήσιμες τεχνικές: \tl{PCA}, \tl{Autoencoders}, \tl{CatBoost}
  \item Ερμηνευσιμότητα με \tl{SHAP}
  \item Υπάρχει πρόκληση στην πρόβλεψη της κατηγορίας \(1\).
\end{itemize}
\end{frame}
\end{document}
